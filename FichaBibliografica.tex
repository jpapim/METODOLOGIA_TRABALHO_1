\documentclass[11pt]{article}
\usepackage[utf8x]{inputenc}
\usepackage[brazil]{babel}
\usepackage[a4paper,top=2cm,bottom=2cm,left=3cm,right=3cm,marginparwidth=1.75cm]{geometry}

\begin{document}
\noindent
\begin{center}
\Large\textbf{Universidade de Brasília}\\
\Large\textbf{Faculdade de tecnologia}\\
\Large\textbf{Departamento de Engenharia Elétrica}\\
\Large\textbf{Programa de Pós-Graduação Profissional \\
	em Engenharia Elétrica -- PPEE}\\
\end{center}

\large{
\noindent
\textbf{Disciplina:} Metodologia de Pesquisa Científica - 2018/2\\
\textbf{Professor:} Rafael Timóteo de Sousa Júnior \\
\textbf{Aluno(a): João Paulo Pimentel}}\\

\begin{center}
\Large\textbf{Ficha Bibliográfica}
\end{center}
\normalsize

\section{Identificação do documento}
\begin{center}
	\begin{tabular}{|l|l|} \hline
		\textbf{Título} & A Filosofia do Não\cite{Bachelard1979} \\ \hline
		\textbf{Autor} & BACHELARD, Gaston \\ \hline
		\textbf{Editor/Distribuidor} & Victor Civita / Abril Cultural
 \\ \hline
		\textbf{Edição} & 2nd ed. \\ \hline
		\textbf{Ano de publicação} & 1979 \\ \hline
		\textbf{Tradução} &  Joaquim José Moura Ramos \\ \hline
	\end{tabular}
\end{center}

\section{Tipo de texto}
    Científico e Filosófico.

\section{Descrição do assunto}
    No início do livro, o autor explica sobre a posição do cientista, bem como a posição do filósofo, deixando bem óbvio suas ligações, levando ao entendimento principal entre o espírito científico e o não científico. A "filosofia do não" é explicada como uma atitude de conciliação entre a observação aberta e a observação fechada. No primeiro capítulo é mostrada como a perspectiva filosófica é evoluída do ultra-racionalismo ao pensamento científico moderno, utilizando a ideia de massa para exemplificação do processo.

\section{Informação selecionada}
    Fichamento sobre o prefácio e 1º capítulo do livro "A Filosofia do Não".
    Excelente leitura para a compreender a ideia do cientista com relação ao objeto e como essa postura evoluiu com o passar dos tempos.

\bibliographystyle{ieeetr}
\bibliography{referencias}
\end{document}
