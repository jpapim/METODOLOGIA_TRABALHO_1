\documentclass[11pt]{article}
\usepackage[utf8]{inputenc}
\usepackage[brazil]{babel}
\usepackage{cite}
\usepackage[a4paper,top=3cm,bottom=2cm,left=3cm,right=3cm,marginparwidth=1.75cm]{geometry}

\begin{document}
\noindent
\begin{center}
	\Large\textbf{Universidade de Brasília}\\
	\Large\textbf{Faculdade de tecnologia}\\
	\Large\textbf{Departamento de Engenharia Elétrica}\\
	\Large\textbf{Programa de Pós-Graduação Profissional \\
		em Engenharia Elétrica -- PPEE}\\
\end{center}

\large{
	\noindent
	\textbf{Disciplina:} Metodologia de Pesquisa Científica - 2018/2\\
	\textbf{Professor:} Rafael Timóteo de Sousa Júnior \\
	\textbf{Aluno(a): João Paulo Pimentel}}\\

\begin{center}
\Large\textbf{Resenha Crítica}
\end{center}
\normalsize

\section{Identificação da obra}
Apresentar os dados bibliográficos essenciais do livro ou artigo objeto da resenha (do mesmo modo que na ficha de leitura).

\begin{center}
	\begin{tabular}{|l|l|} \hline
		\textbf{Título} & A Filosofia do Não\cite{Bachelard1979} \\ \hline
		\textbf{Autor} & Bachelard, Gaston \\ \hline
		\textbf{Editor/Distribuidor} & VICTOR CIVITA
 \\ \hline
		\textbf{Edição} & 2. ed. \\ \hline
		\textbf{Ano de publicação} & 1979 \\ \hline
		\textbf{Tradução} &  Joaquim José Moura Ramos, Remberto Francisco Kuhnen Antônio da Costa Leal e Lídia do Valle Santos Leal \\ \hline
	\end{tabular}
\end{center}

\section{Identificação do autor da obra}
Falar quem é o autor da obra que foi resenhada, tratando brevemente da vida e de algumas outras obras do escritor ou pesquisador.

\section{Apresentação da obra}
Situar o leitor descrevendo em poucas linhas todo o conteúdo do texto resenhado.

Descrever a estrutura: falar sobre a divisão em capítulos, em seções, sobre o foco narrativo ou até, de forma sutil, o número de páginas do texto completo.

Descrever o conteúdo: utilizar de 3 a 5 parágrafos para resumir claramente o texto resenhado.

\section{Análise crítica}
Nesta parte, e apenas nessa parte, dar uma opinião. Argumentar baseando-se em teorias de outros autores, fazendo comparações ou até mesmo utilizando-se de explicações que foram dadas em aula. É difícil encontrar resenhas que utilizam mais de 3 parágrafos para isso, porém não há um limite estabelecido, podendo haver uma extensão criteriosa em função do tamanho do conteúdo resenhado

Recomendar a obra: analisar para quem o texto realmente é útil (se for útil para alguém), utilizando elementos sociais ou pedagógicos, como idade, escolaridade, renda etc.

\section{Autor da resenha}
Resenha elaborada por João Paulo Pimentel, estudante de mestrado do PPEE. 

\bibliographystyle{ieeetr}
\bibliography{referencias}
\end{document}

