\documentclass[11pt]{article}
\usepackage[utf8]{inputenc}
\usepackage[brazil]{babel}
\usepackage{cite}
\usepackage[a4paper,top=3cm,bottom=2cm,left=3cm,right=3cm,marginparwidth=1.75cm]{geometry}

\begin{document}
\noindent
\begin{center}
	\Large\textbf{Universidade de Brasília}\\
	\Large\textbf{Faculdade de Tecnologia}\\
	\Large\textbf{Departamento de Engenharia Elétrica}\\
	\Large\textbf{Programa de Pós-Graduação Profissional \\
		em Engenharia Elétrica -- PPEE}\\
\end{center}

\large{
	\noindent
	\textbf{Disciplina:} Metodologia de Pesquisa Científica - 2018/2\\
	\textbf{Professor:} Rafael Timóteo de Sousa Júnior \\
	\textbf{Aluno(a): João Paulo Pimentel}}\\

\begin{center}
\Large\textbf{Resenha Crítica}
\end{center}
\normalsize

\section{Identificação da obra}
    Apresentar os dados bibliográficos essenciais do livro ou artigo objeto da resenha (do mesmo modo que na ficha de leitura).

\begin{center}
	\begin{tabular}{|l|l|} \hline
		\textbf{Título} & A Filosofia do Não\cite{Bachelard1979} \\ \hline
		\textbf{Autor} & BACHELARD, Gaston \\ \hline
		\textbf{Editor/Distribuidor} & Victor Civita / Abril Cultural
 \\ \hline
		\textbf{Edição} & 2nd ed. \\ \hline
		\textbf{Ano de publicação} & 1979 \\ \hline
		\textbf{Tradução} &  Joaquim José Moura Ramos \\ \hline
	\end{tabular}
\end{center}

\section{Identificação do autor da obra}

\paragraph{}
    O autor da obra é o Gaston Bachelard, que viveu de 1884 até 1962, foi membro da Academia de Ciências Morais e Políticas da França, além de ter ganhado o Prêmio Nacional de Letras, também é autor de várias obras filosóficas, onde vale a pena destacar a considerada obra mais importante da sua vida que foi o Livro "O Racionalismo Aplicado".

\section{Apresentação da obra}

\paragraph{}
    A resenha crítica é referente ao prefácio e ao 1º capítulo do Livro "A Filosofia do Não" de Gaston Bachelard. O livro todo é dividido em 4 partes, a primeira parte é referente a "Vida e Obra" (Cronologia e Bibliografia) do autor, a segunda parte é sobre a "Filosofia do Não" que é subdividida em seis capítulos, já a terceira parte é sobre "O Novo Espírito Científico que é subdividido em seis capítulos, e a quarta é última parte do livro é sobre "A Poética do Espaço" que é subdividido em dez capítulos.

\paragraph{}
    Durante a leitura do prefácio e do 1º capítulo do Livro "A Fisolofia do Não", um dos problemas identificados no texto de Gaston Bachelard é o de que um pensamento filosófico somente deve ser utilizado para a proposta dele assim definida, mesmo com a necessidade da aplicação de uma filosofia fechada a um aforismo ou um pensamento científico aberto. Desta forma, o pensamento do cientista é explicado por desabonar a preparação transcendental e trabalhar com as lições do conhecimento nas ciências experimentais e com os princípios da evidência racional nas ciências exatas. Como contrapeso, a posição dos filósofos é de considerar suficiente a meditação coordenada do pensamento, sem preocupação com as multiplicidades dos fatos.

\paragraph{}
    Segundo os cientistas, a ciência está no domínio dos fatos, ou seja, ela é aberta, pois sempre está incompleta ou inacabada. Já para os filósofos, a camada experimental é fechada e a ciência nunca está totalmente no domínio dos fatos, assim os exemplos científicos são relembrados, mas não esclarecidos. A ideia de Gaston Bachelard no texto traz este interessante entendimento contraponto para concluir que os estudos dos filósofos são aqueles dos princípios muito sucintos e o estudo dos cientistas é aquele dos resultados muito mais explicado e exemplificado, facilitando assim o entendimento.  

\paragraph{}
    Entendendo melhor as ideias do autor fica mais fácil compreender a “filosofia do não” apresentada, pois temos que ser conscientes no sentido que uma experiência recente ou como se diz no texto, uma experiência nova, diz “não” a uma experiência passada, ou antiga. E a lenda científica deve fazer modificações intensas no pensamento e para isto ela deve ser vasta e aberta, diferente da cultura filosófica em que a eficácia muitas vezes está na sua função particular. Já o pensamento científico é um método de dispersão bem ordenado, que na visão atualizada por Gaston Bachelard deve considerar também o subjetivo, numa filosofia dispersa que torna a “filosofia do não” uma atitude de acordo, não de rejeita, aceitando abreviar concomitantemente todo o conhecimento e todo o adágio da determinação de uma substância. Ou seja, o autor defende um método objetivo baseado em evidencias, mas que tem espaço na sua realização para a subjetividade, a apreensão e o domínio coerente.


\section{Análise crítica}

\paragraph{}
    Como análise crítica, vale destacar o que foi descrito no primeiro capítulo do livro “A Filosofia do Não”, pois Gaston Bachelard escolhe explicar uma visão da evolução do pensamento científico, partindo do ultra-racionalismo, que ele entende como a reunião do racionalismo complexo e do racionalismo dialético, para expor a evolução filosófica de um conhecimento científico particular que é um movimento que atravessa todas estas doutrinas na ordem indicada, que, no caso, ele escolhe o conceito de massa para demonstrar a maturação filosófica deste pensamento científico. 

\paragraph{}
    Analisando criticamente o conceito de massa que foi apresentado pelo autor, ele expõe o primeiro entendimento, e ficou bem clara, a noção de massa como concretização do desejo de se alimentar, uma ideia carregada da simbologia da prova adquirida pelos mais antigos, de uma riqueza doméstica, concentrada de estimas. A segunda noção de massa recorre à objetividade instrumental para ligar o conceito à utilização da balança, ou seja, o conceito passa a ser carregado de doutrina segundo a qual todo conhecimento provém unicamente da experiência e já começa a iniciar a caminhada no contexto do pensamento absolutista.
    
\paragraph{}
    Entretanto, o autor apresenta um texto conveniente e tão bem construído que deixa claro a ideia da “Filosofia do não” como um estilo de melhora pelo constrangimento, ao questionar o que está posto e buscar novos entendimentos, além de expor de forma didática a evolução do pensamento filosófico e do espírito científico, que sai do fechado para o aberto, da teoria baseada na experiência íntima para as teorias da filosofia lógica, que molda o novo espírito científico facilmente compreendido por todos que estarão lendo o texto recomendado para alunos de Graduação e Pós-Graduação na área de Filosofia, Pedagogia e áreas afins.

\section{Autor da resenha}
Resenha elaborada por João Paulo Pimentel, estudante de Mestrado Profissional do PPEE. 

\bibliographystyle{ieeetr}
\bibliography{referencias}

	SOUSA JR, R. T. Aula do Professsor de Metodologia de Pesquisa Científica. Apresentação na Faculdade de Tecnologia no Mestrado Profissional - PPEE, UnB. 2º semestre, 2018.  

\end{document}

